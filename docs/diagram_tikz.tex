\documentclass[12pt,oneside,tikz]{standalone}
\usepackage{pgf,tikz,comment}
\usepackage{circuitikz}
\usepackage[active,tightpage]{preview}
\PreviewEnvironment{tikzpicture}
\usetikzlibrary{positioning,calc,arrows}
\usepackage[a4paper,landscape,margin=0.5cm,left=3.75cm]{geometry}
\ctikzset{logic ports=ieee}
\pagestyle{empty}

\begin{document}

\title{Figure}
\author{}
\maketitle
\section{Figure}

\begin{tikzpicture}[>=latex, every node/.style={font=\small\sffamily}, node distance=2.5cm]

\tikzset{flipflop DQ/.style={flipflop, scale=.7,
         flipflop def={t1=D, t6=Q, c3=1, clock wedge size=.3, font=\normalsize},
}}
\tikzset{flipflop DQC/.style={flipflop, scale=.7,
         flipflop def={t1=D, t6=Q, c3=1, nd=1, clock wedge size=.3, font=\normalsize},
}}

% reused flipflop model for ports
\tikzset{counter/.style={
         %draw,rectangle,minimum height=25mm,minimum width=15mm}
         flipflop, scale=1,
         flipflop def={t1=EN, t2=SEL, t6=O, nd=1, clock wedge size=2, font=\normalsize},
}}

\tikzset{linebusticko/.style={
         draw,rectangle,very thin,minimum height=0.5mm,minimum width=0.5mm}
}

\tikzstyle{fonta} = [draw,rectangle,text centered]
\tikzstyle{fontb} = [draw,rectangle,text centered,font=\bf\it]

\newcommand\currentcoordinate{\the\tikz@lastxsaved,\the\tikz@lastysaved}
\newcommand\currentx{\the\tikz@lastxsaved}
\newcommand\currenty{\the\tikz@lastysaved}
\makeatother
\newcommand\fontb{\bf}
%\newcommand\linebustickzzz{\draw let \p1 = (\currentcoordinate) in [thick] (\p1) -- ++(0.1em,0.1em)}
\newcommand\linebustick{\textbf{/}} % cheat with font

\coordinate (origin) at (0,0);
\coordinate (portin) at (0,0); % used for absolute X value
\coordinate (portout) at (20,0); % used for absolute Y value

%\begin{circuitikz} %[nodes=draw]
  \node[counter] (counter) at (3.5,6.75) {};
  \node[clockwedge, scale=1.25] at ([yshift=-8.5mm] counter.west) (counter_clk) {};
  \node[flipflop DQC] (R1) at (10,7) {};
  \node[flipflop DQC] (R2) at (15,7) {};
  \node[flipflop DQC] (Rskew) at (8,2) {};

  \node[above=2pt of R1.north] {\small R1};
  \node[above=2pt of R2.north] {\small R2};
  \node[above=2pt of counter.north] {\small COUNTER};

  % clk
  \draw let
    \p1 = (portin),
    \p2 = (counter.pin 3),
    \p3 = (\x1,\y2)
    in {
        node[anchor=east, align=right, text width=1cm] (text_clk) at ([xshift=-0.1em] \p3) {\fontb{clk}}
        node[circ] (circ_clk) at (\p3) {}
    };
  \draw let
    \p1 = (portin),
    \p2 = (counter.pin 3),
    \p3 = (\x1,\y2)
    in [thick, blue] (\p3) -- node[above, midway] {clk1} (counter.pin 3) -- (counter_clk); % needed extra leg
  \draw[thick, blue] (circ_clk) -| ++(1.5cm,0) |- ++(0,-2cm) |- (Rskew.pin 1);

  % rst_n
  \draw let
    \p1 = (portin),
    \p2 = ([yshift=-1.15cm] circ_clk),
    \p3 = (\x1,\y2)
    in {
        node[anchor=east, align=right, text width=1cm] (text_rst_n) at ([xshift=-0.1em] \p3) {\fontb{rst\_n}}
        node[circ] (circ_rst_n) at (\p3) {}
    };
  \draw[thick,brown] (circ_rst_n) -| (counter-Nd.south); % -Nd offsets bubble
  \draw[thick,brown] (circ_rst_n) -| (R1-Nd.south);
  \draw[thick,brown] (circ_rst_n) -| (R2-Nd.south);
  \draw[thick,brown] (circ_rst_n) -| ++(2.25cm,0) |- ++(0,-4.5cm) -| (Rskew-Nd.south);

  % ui_in0
  \draw let
    \p1 = (portin),
    \p2 = ([below left=5mm of R1,xshift=-5mm,yshift=-33mm] R1),
    \p3 = (\x1,\y2)
    in {
      node[anchor=east, align=right, text width=2cm] (text_ui_in0) at ([xshift=-0.1em] \p3) {\fontb{ui\_in[0]}}
      node[circ] (circ_ui_in0) at (\p3) {}
    };
  \draw[thick, teal] (circ_ui_in0) -- +(12cm,0) node[above, midway] {clk2} -| ([xshift=-3mm] R2.pin 3) |- (R2.pin 3);
  \draw[thick, teal] (circ_ui_in0) -| ([xshift=-3mm] R1.pin 3) |- (R1.pin 3);
  \draw[thick, teal] (Rskew.pin 6) -- ++(3,0);
  \draw[thick, teal] (circ_ui_in0) |- ++(0.75cm,0) |- ++(0,-2.28cm) -| (Rskew.pin 3); % -2.28cm could be better specified

  % ui_in1
  \draw let
    \p1 = (portin),
    \p2 = (counter.pin 1),
    \p3 = (\x1,\y2)
    in {
        node[anchor=east, align=right, text width=2cm] (text_ui_in0) at ([xshift=-0.1em] \p3) {\fontb{ui\_in[2]}}
        node[circ] (circ_ui_in1) at (\p3) {}
    };
  \draw[thick] (circ_ui_in1) -- (counter.pin 1);

  % ui_in2
  \draw let
    \p1 = (portin),
    \p2 = (counter.pin 2),
    \p3 = (\x1,\y2)
    in {
        node[anchor=east, align=right, text width=2cm] (text_ui_in0) at ([xshift=-0.1em] \p3) {\fontb{ui\_in[1]}}
        node[circ] (circ_ui_in2) at (\p3) {}
    };
  \draw[thick] (circ_ui_in2) -- (counter.pin 2);

  \draw[very thick] (counter.pin 6) -- node[above, midway, yshift=0.25em] {$cnt$}
                             node[midway] {\linebustick}
                             node[below, midway, yshift=-0.25em] {$4$}
                             node[below, midway, yshift=-5mm] {stage1}
                             (R1.pin 1);

  \draw[very thick] (R1.pin 6) -- node[above, midway, yshift=0.25em] {$cnt_{\textsubscript{\textsc{w}}}$}
                             node[midway] {\linebustick}
                             node[below, midway, yshift=-0.25em] {$4$}
                             node[below, midway, yshift=-5mm] {stage2}
                             (R2.pin 1);
  % uio_out7
  \draw let 
    \p1 = (portout),
    \p2 = ([yshift=6em] R2.pin 6),
    \p3 = (\x1,\y2)
    in {
      node[anchor=west, text width=2cm] (text_uio_out7) at ([xshift=0.2em] \p3) {\fontb{uio\_out[7:4]}}
      node[circ] (circ_uio_out7) at (\p3) {}
    };
  \draw let
    \p1 = ([xshift=-12cm] circ_uio_out7),
    \p2 = (R1.pin 1)
    in [very thick] (circ_uio_out7) -| (\p1) -- (\x1,\y2) node[circ] {};

  % uio_out3
  \draw let 
    \p1 = (portout),
    \p2 = ([yshift=3em] R2.pin 6),
    \p3 = (\x1,\y2)
    in {
      node[anchor=west, text width=2cm] (text_uio_out3) at ([xshift=0.2em] \p3) {\fontb{uio\_out[3:0]}}
      node[circ] (circ_uio_out3) at (\p3) {}
    };
  \draw let
    \p1 = ([xshift=-6.5cm] circ_uio_out3),
    \p2 = (R2.pin 1)
    in [very thick] (circ_uio_out3) -| (\p1) -- (\x1,\y2) node[circ] {};

  % uo_out3
  \draw let 
    \p1 = (portout),
    \p2 = (R2.pin 6),
    \p3 = (\x1,\y2)
    in [very thick] (R2.pin 6) -- node[above, midway, yshift=0.25em] {$cnt_{\textsubscript{\textsc{s}}}$}
                             node[midway] {\linebustick}
                             node[below, midway, yshift=-0.25em] {$4$}
                             node[below, midway, yshift=-5mm] {\small stage3}
                             (\p3) node[circ] {}
                             node[anchor=west, text width=2cm] (text_uo_out3) at ([xshift=0.2em] \p3) {\fontb{uo\_out[3:0]}};

  % uo_out4
  \draw let
    \p1 = (portout),
    \p2 = (Rskew.pin 6),
    \p3 = (\x1,\y2)
    in {
      node[anchor=west, text width=2cm] (text_uo_out4) at ([xshift=0.2em] \p3) {\fontb{uo\_out[4]}}
      node[circ] (circ_uo_out4) at (\p3) {}
    };
  \draw let
    \p1 = (portout),
    \p2 = (Rskew.pin 6),
    \p3 = (\x1,\y2)
    in [thick] (Rskew.pin 6) -- node[above, midway] {$skew$} (\p3);

  % uo_out7
  \node[thick, ground] (ground) at ([xshift=-0.5cm, yshift=1cm] portout) {};
  \draw let
    \p1 = (portout),
    \p2 = ([yshift=-1.5em] circ_uo_out4),
    \p3 = (\x1,\y2)
    in [thick] (ground) |- (\p3);
  \draw let
    \p1 = (portout),
    \p2 = ([yshift=-1.5em] circ_uo_out4),
    \p3 = (\x1,\y2)
    in {
      node[anchor=west, text width=2cm] (text_uo_out7) at ([xshift=0.2em] \p3) {\fontb{uo\_out[7:5]}}
      node[circ] (circ_uo_out7) at (\p3) {}
    };

  \node[draw, align=center, text width=20cm] at ([xshift=-3em, yshift=3em] $(current page.south)$)
    {\textbf{Based on Figure 29.3 pg580, Digital Design: A Systems Approach, ISBN 0521199506}};

  %\draw (-1,0.25) -- (5,0.25);

  \draw (0,0) -- (0,0.5);
  \draw (0,0) -- (0,-1);
  \draw (0,0) -- (0.5,0);
  \draw (0,0) -- (-1,0);
%\end{circuitikz}

\begin{comment}
% all comments here are treated as comments
\end{comment}

\end{tikzpicture}

\clearpage
\end{document}
